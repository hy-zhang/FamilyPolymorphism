%-----------------------------------------------------------------------------
%
%               Template for sigplanconf LaTeX Class
%
% Name:         sigplanconf-template.tex
%
% Purpose:      A template for sigplanconf.cls, which is a LaTeX 2e class
%               file for SIGPLAN conference proceedings.
%
% Guide:        Refer to "Author's Guide to the ACM SIGPLAN Class,"
%               sigplanconf-guide.pdf
%
% Author:       Paul C. Anagnostopoulos
%               Windfall Software
%               978 371-2316
%               paul@windfall.com
%
% Created:      15 February 2005
%
%-----------------------------------------------------------------------------


\documentclass[preprint,numbers,nocopyrightspace]{sigplanconf}

% The following \documentclass options may be useful:

% preprint       Remove this option only once the paper is in final form.
%  9pt           Set paper in  9-point type (instead of default 10-point)
% 11pt           Set paper in 11-point type (instead of default 10-point).
% numbers        Produce numeric citations with natbib (instead of default author/year).
% authorversion  Prepare an author version, with appropriate copyright-space text.

\usepackage{amsmath}
\usepackage{listings}

% Remote packages

% For pdflatex, replaced by fontspec:
% \usepackage{tgpagella}
\usepackage[T1]{fontenc}
\usepackage[utf8]{inputenc}

% For xelatex or lualatex:
% \usepackage{fontspec}
% \setmainfont{Times New Roman}


\usepackage{amsmath}
\usepackage{amsthm}
\usepackage{amssymb}
\usepackage{mathtools} % For \Coloneqq
\usepackage{bm}        % Bold symbols in maths mode
\usepackage{cmll}
\usepackage{fixltx2e}
\usepackage{stmaryrd}
\usepackage[dvipsnames]{xcolor}
\usepackage{listings} % For code listings
% \usepackage{minted}
% \usemintedstyle{murphy}
\usepackage{fancyvrb}
\usepackage{url}
\usepackage{xspace}
\usepackage{comment}
\usepackage{mdwlist}

% Typography
\usepackage[euler-digits,euler-hat-accent]{eulervm}

% Copied from the FCore paper:
\usepackage[colorlinks=true,allcolors=black,breaklinks,draft=false]{hyperref}   % hyperlinks, including DOIs and URLs in bibliography
% known bug: http://tex.stackexchange.com/questions/1522/pdfendlink-ended-up-in-different-nesting-level-than-pdfstartlink

% Figures with borders
% http://en.wikibooks.org/wiki/LaTeX/Floats,_Figures_and_Captions
% \usepackage{float}
% \floatstyle{boxed}
% \restylefloat{figure}

% Local packages

% \usepackage{styles/bcprules}    % by Benjamin C. Pierce
\usepackage{styles/mathpartir}  % by Didier Rémy


% ! Always load mathastext last
% http://mirrors.ibiblio.org/CTAN/macros/latex/contrib/mathastext/mathastext.pdf
% \renewcommand\familydefault\ttdefault
% \usepackage{mathastext}
% \renewcommand\familydefault\rmdefault

% http://tex.stackexchange.com/questions/114151/how-do-i-reference-in-appendix-a-theorem-given-in-the-body
\usepackage{thmtools, thm-restate}
\newtheorem{theorem}{Theorem}
\newtheorem{lemma}{Lemma}

\newcommand{\turns}{\vdash}

\newcommand{\im}[1]{\lvert #1 \rvert}

\newcommand{\hast}{\!:\!}
\newcommand{\subst}[2]  {\lbrack #1 / #2 \rbrack}

% Relations
\newcommand{\subtype}   {<:}

\definecolor{facebook}{HTML}{3B5998}
\newcommand{\yields}[1]{\textcolor{facebook}{\; \hookrightarrow {#1}}}

% Helpers
\newcommand{\ftv}[1]{\textit{ftv}({#1})}

% Spacing
\newcommand{\binderspacing}{\,}
\newcommand{\appspacing}{\;}

% Types
\newcommand{\for}[2]{\forall #1. \binderspacing #2}
\newcommand{\recty}[2]{\{ #1 \hast #2 \}}
% \newcommand{\top}{\{\}}
\newcommand{\andop}{\with}
\newcommand{\pair}[2]{(#1, #2)}

% Expressions
\newcommand{\lam}[3]{\lambda (#1 \hast #2).\binderspacing #3}
\newcommand{\blam}[2]{\Lambda #1.\binderspacing #2}
\newcommand{\app}[2]{#1 \appspacing #2}
\newcommand{\tapp}[2]{#1 \appspacing #2}
\newcommand{\mergeop}{,,}
\newcommand{\reccon}[2]{\{ #1 = #2 \}}
\newcommand{\recupdate}[3]{#1 \; \mathbf{with} \; \{#2 = #3\}}
\newcommand{\proj}[2]{{\code{proj}}_{#1} #2}
\newcommand{\letexpr}[3]{\kwlet \; #1 = #2 \; \kwin \; #3}

% Keywords
\newcommand{\keyword}[1]{\texttt{#1}}

\newcommand{\kwlet}{\keyword{let}}
\newcommand{\kwin}{\keyword{in}}
\newcommand{\kwwhere}{\keyword{where}}


\newcommand{\Int}{\code{Int}}
\newcommand{\String}{\code{String}}
\newcommand{\Bool}{\code{Bool}}
\newcommand{\I}{\code{i}}
\newcommand{\J}{\code{j}}


% Rules

% Couleurs
\colorlet{subcolor}{OliveGreen}
\colorlet{targetcolor}{BrickRed}

% Source/elaboration and labels
\newcommand{\rulelabelerecupd}{\rulelabele\text{rec-upd}}


% Presentation
\definecolor{lightyellow}{HTML}{FFFFE0}
\newcommand{\highlight}[1]{\colorbox{GreenYellow}{$#1$}}


% To be retired
\newcommand{\turnsget}{\vdash_{\textrm{get}}}
\newcommand{\turnsput}{\vdash_{\textrm{put}}}
\newcommand{\turnsrec}{\vdash_{\textrm{rec}}}
\newcommand{\rulename}[1]{(\textrm{#1})}


%%%%%%%%%%%%%%%%%%%%%%%%%%%%%%%%%%%%%%%%%% syntax.tex begin %%%%%%%%%%%%%%%%%%%%%%
\newcommand{\judgeewf}[2]{#1 \; {\turns} \; #2}

\newcommand{\kwinterface}{\keyword{interface}}
\newcommand{\kwextends}{\keyword{extends}}
\newcommand{\kwreturn}{\keyword{return}}
\newcommand{\kwoverride}{\keyword{override}}
\newcommand{\kwsuper}{\keyword{super}}
\newcommand{\kwthis}{\keyword{this}}
\newcommand{\kwnew}{\keyword{new}}
\newcommand{\kwtrue}{\keyword{true}}
\newcommand{\kwfalse}{\keyword{false}}


\newcommand{\mtype}{\keyword{mtype}}
\newcommand{\ext}{\keyword{ext}}
\newcommand{\definedin}{\keyword{definedin}}
\newcommand{\collectMethods}{\keyword{collectMethods}}
\newcommand{\mbody}{\keyword{mbody}}
\newcommand{\Undefined}{\keyword{Undefined}}
\newcommand{\Error}{\keyword{Error}}
\newcommand{\needed}{\keyword{needed}}
\newcommand{\methods}{\keyword{methods}}
\newcommand{\only}{\keyword{only}}

\newcommand{\new}[1]{
    \kwnew \; #1()
}

\newcommand{\interface}[3]{
  \kwinterface \; #1 \; \kwextends \; \overline{#2} \; {\{} \overline{#3} {\}}
}
\newcommand{\method}[6]{
  #1 \; #2 (\overline{#3} \; #4) \; \kwoverride \; #5 \; {\{} \kwreturn \; #6 ; {\}}
}

\newcommand{\subid} {
\inferrule* [right=]
    {}
    {I \subtype I}
}
\newcommand{\subtrans} {
\inferrule* [left=]
    {I \subtype J \\ J \subtype K}
    {I \subtype K}
}

\newcommand{\subextends} {
\inferrule* [right=]
    {\interface{I}{I}{M}}
    {\forall I_i \in \overline{I}, I \subtype I_i}
}

\newcommand{\tvar} {
\inferrule* [left=(T-Var)]
    {}
    {\judgeewf \Gamma x:\Gamma(x)}
}

\newcommand{\tinvk} {
\inferrule* [left=(T-Invk)]
    {  \judgeewf \Gamma {e_0:I_0}
    \\ \mtype(m, I_0) = \overline{J} \to I
    \\ \judgeewf \Gamma \overline{e}:\overline{I}
    \\ \overline{I} \subtype \overline{J}
    }
    {\judgeewf \Gamma e_0.m(\overline{e}):I}
}

\newcommand{\tpathinvk} {
\inferrule* [left=(T-PathInvk)]
    {  \judgeewf \Gamma {e_0:I_0}
    \\ I_0 \subtype J_0
    \\ \mtype(m, J_0) = \overline{J} \to I
    \\ \judgeewf \Gamma \overline{e}:\overline{I}
    \\ \overline{I} \subtype \overline{J}
    }
    {\judgeewf \Gamma e_0.J_0::m(\overline{e}):I}
}

\newcommand{\tsuperinvk} {
\inferrule* [left=(T-SuperInvk)]
    {  \judgeewf \Gamma {this:I_0}
    \\ \ext(I_0, J_0)
    \\ \mtype(m, J_0) = \overline{J} \to I
    \\ \judgeewf \Gamma \overline{e}:\overline{I}
    \\ \overline{I} \subtype \overline{J}
    }
    {\judgeewf \Gamma_{I_0} {\kwsuper.J_0::m(\overline{e})}:I}
}

\newcommand{\tnew} {
\inferrule* [left=(T-New)]
    {}
    {\judgeewf \Gamma \new{I}:I}
}

\newcommand{\tmethod} {
\inferrule* [left=(T-Method)]
    {  \ext(I, J)
    \\ \mtype(m, J) = \overline{I} \to I_0
    \\ \text{If } I=J \text{ then } \only(m, I) = \kwtrue
    %\\ \definedin(m, J) % m is (directly or indirectly) defined in J
    }
    {\method{I_0}{m}{I}{x}{J}{e_0} \text{ OK IN } I}
}

\newcommand{\tintf} {
\inferrule* [left=(T-Intf)]
    {  \overline{I} \text{ OK}
    \\ \forall m \in \collectMethods(I), \mbody(m, I) \neq \Undefined
    }
    { \interface{I}{I}{M} \text{ OK }}
}

\newcommand{\sinvk} {
\inferrule* [left=(S-Invk)]
{\mbody(m, I, J) = (\overline{X} \; \overline{x}, E' \; e_0)}
{<J>\new{I}.m(<\overline{E}>\overline{e}) \to
    <E'>[<\overline{X}>\overline{e}/\overline{x}, <J>\new{I}/\kwthis]e_0}
}

\newcommand{\spathinvk} {
\inferrule* [left=(S-PathInvk)]
{\mbody(m, I, K) = (\overline{X} \; \overline{x}, E' \; e_0)}
{<J>\new{I}.K::m(<\overline{E}>\overline{e}) \to
    <E'>[<\overline{X}>\overline{e}/\overline{x}, <J>\new{I}/\kwthis]e_0}
}

\newcommand{\ssuperinvk} {
\inferrule* [left=(S-SuperInvk)]
{\mbody(m, K, K) = (\overline{X} \; \overline{x}, E' \; e_0)}
{\kwsuper.K::m(<\overline{E}>\overline{e}) \to
    <E'>[<\overline{X}>\overline{e}/\overline{x}, <J>\new{I}/\kwthis]e_0}
}

\newcommand{\deff} {
\begin{displaymath}
    \begin{array}{l}
        \begin{array}{llrl}
        \text{Annotated Expressions}   & f & \Coloneqq & e \mid <I>e
        \end{array}
    \end{array}
\end{displaymath}
}

\newcommand{\creceiver} {
%----------- v1.0 ---------------
%\inferrule* [left=(C-Receiver)]
%{e \to e'}
%{e.m(\overline{e}) \to e'.m(\overline{e})}
%}
\inferrule* [left=(C-Receiver)]
{f \to f'}
{f.m(\overline{e}) \to f'.m(\overline{e})}
}

\newcommand{\cpathreceiver} {
\inferrule* [left=(C-PathReceiver)]
{f \to f'}
{f.K::m(\overline{e}) \to f'.K::m(\overline{e})}
}

\newcommand{\cargs} {
%---------- v1.0 ---------------
%\inferrule* [left=(C-Args)]
%{e_i \to e_i'}
%{e.m(..., e_i, ...) \to e.m(..., e_i', ...)}
%}
%---------- v2.0 ---------------
%\inferrule* [left=(C-Args)]
%{e_1 \to f}
%{<J>\new{I}.m(<\overline{E}>\overline{e_0}, e_1, \overline{e_2})
%\to
%<J>\new{I}.m(<\overline{E}>\overline{e_0}, f, \overline{e_2})}
%}
%---------- v3.0 ---------------
\inferrule* [left=(C-Args)]
{e_1 \to f}
{<J>\new{I}.m(<\overline{E}>\overline{v}, e_1, \overline{e_2})
\to
<J>\new{I}.m(<\overline{E}>\overline{v}, f, \overline{e_2})}
}

\newcommand{\cpathargs} {
\inferrule* [left=(C-PathArgs)]
{e_1 \to f}
{<J>\new{I}.K::m(<\overline{E}>\overline{v}, e_1, \overline{e_2})
\to
<J>\new{I}.K::m(<\overline{E}>\overline{v}, f, \overline{e_2})}
}

\newcommand{\csuperargs} {
\inferrule* [left=(C-SuperArgs)]
{e_1 \to f}
{super.K::m(<\overline{E}>\overline{v}, e_1, \overline{e_2})
\to
super.K::m(<\overline{E}>\overline{v}, f, \overline{e_2})}
}

\newcommand{\cstatictype} {
\inferrule* [left=(C-StaticType)]
{}
{\new{I} \to <I>\new{I}}
}

\newcommand{\cfreduce} {
\inferrule* [left=(C-FReduce)]
{f \to f' \; f \neq \new{I}}
{<I>f \to <I>f'}
}

\newcommand{\cannoreduce} {
\inferrule* [left=(C-AnnoReduce)]
{}
{<I>(<J>e) \to <I>e}
}
%%%%%%%%%%%%%%%%%%%%%%%%%%%%%%%%%%%%%%%%%% syntax.tex end %%%%%%%%%%%%%%%%%%%%%%%%


\newcommand{\cL}{{\cal L}}
\lstset{ %
	language=Java,                % choose the language of the code
	columns=flexible,
	lineskip=-1pt,
	basicstyle=\ttfamily\small,       % the size of the fonts that are used for the code
	numbers=none,                   % where to put the line-numbers
	numberstyle=\ttfamily\tiny,      % the size of the fonts that are used for the line-numbers
	stepnumber=1,                   % the step between two line-numbers. If it's 1 each line will be numbered
	numbersep=5pt,                  % how far the line-numbers are from the code
	backgroundcolor=\color{white},  % choose the background color. You must add \usepackage{color}
	showspaces=false,               % show spaces adding particular underscores
	showstringspaces=false,         % underline spaces within strings
	showtabs=false,                 % show tabs within strings adding particular underscores
	morekeywords={var},
	%  frame=single,                   % adds a frame around the code
	tabsize=2,                  % sets default tabsize to 2 spaces
	captionpos=none,                   % sets the caption-position to bottom
	breaklines=true,                % sets automatic line breaking
	breakatwhitespace=false,        % sets if automatic breaks should only happen at whitespace
	title=\lstname,                 % show the filename of files included with \lstinputlisting; also try caption instead of title
	escapeinside={(*}{*)},          % if you want to add a comment within your code
	keywordstyle=\ttfamily\bfseries,
	aboveskip=0pt,
	belowskip=0pt
	% commentstyle=\color{Gray},
	% stringstyle=\color{Green}
}

\newcommand{\name}{\lstinline{@Family}}
\newcommand{\red}[1]{\textcolor{red}{#1}}

\begin{document}

\special{papersize=8.5in,11in}
\setlength{\pdfpageheight}{\paperheight}
\setlength{\pdfpagewidth}{\paperwidth}

\conferenceinfo{CONF'yy}{Month d--d, 20yy, City, ST, Country}
\copyrightyear{20yy}
\copyrightdata{978-1-nnnn-nnnn-n/yy/mm}\reprintprice{\$15.00}
\copyrightdoi{nnnnnnn.nnnnnnn}

% For compatibility with auto-generated ACM eRights management
% instructions, the following alternate commands are also supported.
%\CopyrightYear{2016}
%\conferenceinfo{CONF'yy,}{Month d--d, 20yy, City, ST, Country}
%\isbn{978-1-nnnn-nnnn-n/yy/mm}\acmPrice{\$15.00}
%\doi{http://dx.doi.org/10.1145/nnnnnnn.nnnnnnn}

% Uncomment the publication rights used.
%\setcopyright{acmcopyright}
\setcopyright{acmlicensed}  % default
%\setcopyright{rightsretained}

\titlebanner{banner above paper title}        % These are ignored unless
\preprintfooter{short description of paper}   % 'preprint' option specified.

\title{Title Text\titlenote{with optional title note}}
\subtitle{Subtitle Text, if any\titlenote{with optional subtitle note}}

\authorinfo{Name1\thanks{with optional author note}}
           {Affiliation1}
           {Email1}
\authorinfo{Name2 \and Name3\thanks{with optional author note}}
           {Affiliation2/3}
           {Email2/3}

\maketitle

\begin{abstract}
This is the text of the abstract.
\end{abstract}

% 2012 ACM Computing Classification System (CSS) concepts
% Generate at 'http://dl.acm.org/ccs/ccs.cfm'.
\begin{CCSXML}
<ccs2012>
<concept>
<concept_id>10011007.10011006.10011008</concept_id>
<concept_desc>Software and its engineering~General programming languages</concept_desc>
<concept_significance>500</concept_significance>
</concept>
<concept>
<concept_id>10003752.10010124.10010138.10010143</concept_id>
<concept_desc>Theory of computation~Program analysis</concept_desc>
<concept_significance>300</concept_significance>
</concept>
</ccs2012>
\end{CCSXML}

\ccsdesc[500]{Software and its engineering~General programming languages}
\ccsdesc[300]{Theory of computation~Program analysis}
% end generated code

% Legacy 1998 ACM Computing Classification System categories are also
% supported, but not recommended.
%\category{CR-number}{subcategory}{third-level}[fourth-level]
%\category{D.3.0}{Programming Languages}{General}
%\category{F.3.2}{Logics and Meanings of Programs}{Semantics of Programming Languages}[Program analysis]

\keywords
keyword1, keyword2

\section{Introduction}\label{sec:introduction}

\section{Overview}\label{sec:overview}

\section{Code Generation}\label{sec:codegeneration}

In this section, we present an overview of the code that \lstinline{@Family} generates. The syntax and type system are
consistent with the Java language. We use translation functions to illustrate our code generation.
To make it clearer, we split the process of code generation into three parts, in which case we introduce two
new annotations \lstinline{@Obj} and \lstinline{@ObjOf}, which are not implemented in the source code but just help to explain
the process of translation. In our implementation, the annotation processing is a combination of the three parts, corresponding to
\lstinline{@Family}, \lstinline{@Obj} and \lstinline{@ObjOf} respectively. The translation rules are presented in Figure \ref{fig:trans1}
and Figure \ref{fig:trans2}, together with some auxiliary functions. Note that in practice, our \textsf{@Family} supports simple
generics without bounds, but for simplicity reason it is not included in the formalization.

\begin{figure}
\begin{tabular}{|l|}
\hline
\begin{lstlisting}
@Family interface New extends Base {
	interface Circuit { int width(); }
	interface Beside {
		default int width() {
			return _c1().width() + _c2().width();
		}
	}
}
\end{lstlisting} \\
\hline
$\Downarrow$ Family desugaring \\
\hline
\begin{lstlisting}
@Obj interface New extends Base {
	@Obj interface Circuit extends Base.Circuit {
		int width();
	}
	@Obj interface Beside extends
			Circuit, Base.Beside {
		Circuit _c1(); Circuit _c2();
		default int width() {
			return _c1().width() + _c2().width();
		}
	}
}
\end{lstlisting} \\
\hline
$\Downarrow$ Obj desugaring \\
\hline
\begin{lstlisting}
@ObjOf interface New extends Base {
	@ObjOf interface Circuit extends Base.Circuit {
		int width();
	}
	@ObjOf interface Beside extends
			Circuit, Base.Beside {
		Circuit _c1(); Circuit _c2();
\end{lstlisting}\vspace{-.05in}\\ \begin{lstlisting}[basicstyle=\ttfamily\small\color{red}]
		Beside withC1(Circuit val);
\end{lstlisting}\\ \begin{lstlisting}
		default int width() {
			return _c1().width() + _c2().width();
		}
	}
}
\end{lstlisting} \\
\hline
$\Downarrow$ ObjOf desugaring \\
\hline
\begin{lstlisting}
interface New extends Base {
	interface Circuit extends Base.Circuit {
		int width();
	}
	interface Beside extends Circuit, Base.Beside {
		Circuit _c1(); Circuit _c2();
		Beside withC1(Circuit val);
		default int width() {
			return _c1().width() + _c2().width();
		}
		static Beside of(Circuit _c1, Circuit _c2) {
			return new Beside() {
				Circuit c1 = _c1; Circuit c2 = _c2;
				public Circuit _c1() { return c1; }
				public Circuit _c2() { return c2; }
				public Beside withC1(Circuit val) {
					return Beside.of(val, this._c2());
				}
			};
		}
	}
	static New of() { return new New(){}; }
}
\end{lstlisting} \\
\hline
\end{tabular}
\caption{An example showing the flow of translation.}\label{fig:flow}
\end{figure}

\subsection{Flow of Translation}
Figure~\ref{fig:flow} illustrates the flow of translation using a simplified version of the Scans example, with intermediate results of desugaring shown. A base family \textsf{Base} is predefined as follows:
\begin{lstlisting}
interface Base {
	interface Circuit {}
	interface Beside extends Circuit {
		Circuit _c1(); Circuit _c2();
		Beside withC1(Circuit val);
	}
}
\end{lstlisting}
\textsf{Base} contains two members, \textsf{Circuit}, and its subtype \textsf{Beside}. \textsf{Beside} has two field methods \textsf{\_c1()} and \textsf{\_c2()} together with a wither method \textsf{withC1}. At this time, another interface \textsf{New} is defined as the new family that extends the old one, where we add an operation \textsf{width()} to the member types. This is precisely the interface that we are going to translate. In the client code, the user simply re-declares the two members with the addition of \textsf{width()}, without building subtyping relations among new and old members, and a simple annotation \textsf{@Family} will deal with all the stuff.

In the first step of translation, namely family desugaring, \textsf{@Family} collects all the members from old families, and fills in the ``extends'' to build the inheritance relations. At the same time, it re-declares all the field methods, for example, although the type name of \textsf{\_c1()} is still \textsf{Circuit}, it refers to the new member type, where we address field type refinements, and hence this new \textsf{\_c1()} has the operation \textsf{width()}. Then it delegates the work to \textsf{@Obj}. In the second phase, \textsf{@Obj} detects fluent setters and withers from the supertypes, then refines their return types to have consistency with the annotated type. In our example, the wither method \textsf{withC1} is refined to have the new \textsf{Beside} as its return type. In the last step of translation, \textsf{@ObjOf} generates constructor methods for all the interfaces, except \textsf{Circuit}, where its \textsf{width()} is an abstract method and we cannot provide a default implementation for it. In the following subsections we will respectively explain the three steps of translation in detail.

\subsection{\textsf{@Family}}
\lstinline{@Family} builds the structure of family polymorphism for the annotated type. More specifically, \lstinline{@Family} tackles two
tasks: (1) building the dependencies (subtyping relations) between new family members and old ones; (2) refining field types. To this
aim, \lstinline{@Family} re-declares all member types and field methods from the base families (see \textsf{collectMembers} and \textsf{fieldMethods} in Figure \ref{fig:trans1}). \lstinline{@Family} recognizes field methods by checking if they are non-void, no-argument methods, and if the name starts with an underscore ``\_''.  The function \textsf{collectMembers} finds all the direct member types from base families $I_1,\cdots,I_n$, creates new types with the same names for re-declaration, and builds the subtyping relations among them. The newly created types are also annotated with \lstinline{@Family}, leading to such a recursive process throughout the nested interfaces in base families. If users have already declared these member types in order to introduce new operations, behaviors or whatever, the function \textsf{combine} will integrate them to the re-declaration, by combining methods and supertypes following ``extends''. Note that we use $\textsf{ibody(}I_m$\textsf{)} to find the exact declaration body of interface $I_m$. \textcolor{red}{Haoyuan: Acutally in the implementation, \textsf{ibody} takes two arguments, one is the name of type reference (interface name $I_m$), and the other is the environment, or where the type reference $I_m$ locates.}

\subsection{\textsf{@Obj}}
The second part of annotation processing is abstracted here using the newly introduced annotation
\lstinline{@Obj}. The translation of \lstinline{@Obj} is like a preprocessing of the third part \lstinline{@ObjOf},
which generates constructor method \textsf{of} for the annotated type. Besides the generation of constructors, \lstinline{@ObjOf} also supports
a series of field and non-field methods, including getters, void and fluent setters, \textsf{with}- methods and the general \textsf{with} method.
\textcolor{red}{Haoyuan: Details on these operations? Can draw a table.} Among them, the fluent setters, \textsf{with}- methods and the general \textsf{with()} method all take the annotated type as their return types, and these types can be refined in some subtypes of the annotated type. What \lstinline{@Obj} precisely does is that it collects all these refinable methods from the super interfaces, and refines them to the annotated type, so that \lstinline{@ObjOf} will later provide a default implementation for them during the generation of constructor methods.

In Figure \ref{fig:trans2}, the first translation function presents the processing of \lstinline{@Obj}. Note that we use $\textsf{mbody(}m,I\textsf{)}$ to get the full declaration of method $m$ seen by $I$, such a method can either be directly defined in $I$ or inherited from its supertypes. Besides delegating \lstinline{@Obj} to the member types in a recursive way, it invokes the function \textsf{refine} to handle the type refinements. The definition of \textsf{refine} is shown in (1) and (2), where return types of \textsf{with-} methods and fluent setters are refined respectively. They are both state operations, and for simplicity we omit the case for general \textsf{with} method in our translation rules.

\subsection{\textsf{@ObjOf}}
The third part of translation relies on \textsf{@ObjOf}, which directly generates the static \textsf{of} method, serving as a constructor method. So it returns an instance of the annotated type. This method takes one argument for each field method from the domain of the interface. Similarly, it recursively generates \lstinline{of} methods for the annotated interface and all nested interfaces.

The translation function of \textsf{@ObjOf} is shown in the second rule of Figure \ref{fig:trans2}. It firstly uses \textsf{valid} to check whether all the abstract methods in the interface are valid; that is to say, any one of them is either a field (getter) method, a \textsf{with-} method or a setter method. We capture these patterns so that \textsf{@ObjOf} can provide default implementations for them. Then, if \textsf{valid} is satisfied, a static \textsf{of} method is generated in the interface to return an instance of it. Such an instance is implemented in the return statement using an anonymous class with auto-generated
implementations for all the methods it supports (stated above). On the other hand, users are expected to put underscores as the
prefix of field methods, and consequently \lstinline{of} identifies these field methods and takes them as its arguments.

\begin{figure*}
\begin{lstlisting}
  (*$\llbracket$*)@Family interface (*$I_m$*) extends (*$I_1,\cdots,I_n$*) {(*$\overline{meth}$*) (*$\overline{I}$*)}(*$\rrbracket$*) = (*$\llbracket$*)@Obj interface (*$I_m$*) extends (*$I_1,\cdots,I_n$*) {(*$\overline{meth'}$*) (*$\overline{I'}$*)}(*$\rrbracket$*)
\end{lstlisting}
\hspace{.3in}where $\overline{meth'}=\overline{meth}\ \cup\ $\textsf{fieldMethods(}$I_1$\textsf{)}$\ \cup\cdots\cup\ $\textsf{fieldMethods(}$I_n$\textsf{)}, $\overline{I'}$ = \textsf{newChilds(}$I_1,\cdots,I_n,\overline{I}$\textsf{)}
~\\~\\
(1) $\overline{I'}$ = \textsf{newChilds(}$I_1,\cdots,I_n,\overline{I}$\textsf{)} \textcolor{red}{(definition of newChilds)}
    \begin{itemize}
    \item $\forall I_0\in\overline{I}$, if $\exists I'_0\in\ $\textsf{collectMembers(}$I_1,\cdots,I_n$\textsf{)} and \textsf{name(}$I_0$\textsf{)} = \textsf{name(}$I'_0$\textsf{)}, then \textsf{combine(}$I_0,I'_0$\textsf{)}$\ \in\overline{I'}$
    \item Otherwise, $I_0\in\overline{I'}$
    \end{itemize}
(2) $\llbracket$\lstinline{@Family}\textsf{ interface }$I$\textsf{ extends }$\overline{I_s},\ \overline{I_t}$\textsf{ \{\}}$\rrbracket\ \in$ \textsf{collectMembers(}$I_1,\cdots,I_n$\textsf{)} \textcolor{red}{(definition of collectMembers)}
    \begin{itemize}
    \item $\forall i,$ s.t. $I\in$\textsf{ childs(}$I_i$\textsf{)},
        \begin{itemize}
        \item $\overline{I_s}$ = \textsf{suptypes(}$I_i.I$\textsf{)}
        \item $I_i.I\in\overline{I_t}$
        \end{itemize}
    \end{itemize}
(3) $\overline{I_0}$ = \textsf{childs(}$I$\textsf{)}, $\overline{I}$ = \textsf{suptypes(}$I$\textsf{)} \textcolor{red}{(definition of childs and suptypes)}
    \begin{itemize}
    \item \textsf{ibody(}$I$\textsf{)} = \textsf{interface }$I$\textsf{ extends }$\overline{I}$\textsf{ \{}$\overline{meth}\ \overline{I_0}$\textsf{\}}
    \end{itemize}
(4) $meth\in\ $\textsf{fieldMethods(}$I$\textsf{)} \textcolor{red}{(definition of fieldMethods)}
    \begin{itemize}
    \item $meth\in\ $\textsf{childs(}$I$\textsf{)}
    \item $meth$ = $I_0$\textsf{ \_m();}
    \end{itemize}
(5) \textsf{interface }$I$\textsf{ extends }$\overline{I_s}$\textsf{ \{}$\overline{meth}\ \overline{I}$\textsf{\}} = \textsf{combine(}$I_m,I_n$\textsf{)} \textcolor{red}{(definition of combine)}
    \begin{itemize}
    \item \textsf{ibody(}$I_m$\textsf{)} = \textsf{interface }$I$\textsf{ extends }$\overline{I_{s1}}$\textsf{ \{}$\overline{meth_1}\ \overline{I_1}$\textsf{\}}
    \item \textsf{ibody(}$I_n$\textsf{)} = \textsf{interface }$I$\textsf{ extends }$\overline{I_{s2}}$\textsf{ \{}$\overline{meth_2}\ \overline{I_2}$\textsf{\}}
    \item $\overline{I_s}$ = $\overline{I_{s1}}\ \cup\ \overline{I_{s2}}$
    \item $\overline{meth}$ = $\overline{meth_1}\ \cup\ \overline{meth_2}$
    \item If $\exists I_1\in\overline{I_1}, I_2\in\overline{I_2}$, \textsf{name(}$I_1$\textsf{)} = \textsf{name(}$I_2$\textsf{)}, then \textsf{combine(}$I_1,I_2$\textsf{)}$\ \in\overline{I}$
    \item Otherwise $(I\in I_1\ \Delta\ I_2)$, $I\in\overline{I}$.
    \end{itemize}
\caption{Translation of \lstinline{@Family}.}
\label{fig:trans1}
\end{figure*}

\begin{figure*}
\begin{lstlisting}
  (*$\llbracket$*)@Obj interface (*$I_0$*) extends (*$\overline{I_s}$*) {(*$\overline{meth}$*) (*$\overline{I}$*)}(*$\rrbracket$*) = (*$\llbracket$*)@ObjOf interface (*$I_0$*) extends (*$\overline{I_s}$*) {(*$\overline{meth}$*) (*$\overline{meth'}$*) (*$\overline{\llbracket\textsf{@Obj}\ I\rrbracket}$*)}(*$\rrbracket$*)
\end{lstlisting}
\hspace{.3in}where $\overline{meth'}$ = \textsf{refine(}$I_0,\overline{meth}$\textsf{)}
\begin{lstlisting}
  (*$\llbracket$*)@ObjOf interface (*$I_0$*) extends (*$\overline{I_s}$*) {(*$\overline{meth}$*) (*$\overline{I}$*)}(*$\rrbracket$*) = interface (*$I_0$*) extends (*$\overline{I_s}$*) {(*$\overline{meth}$*) ofMethod((*$I_0$*)) (*$\overline{I}$*)}
\end{lstlisting}
\hspace{.3in}with \textsf{valid(}$I_0$\textsf{)}, \textsf{of} $\notin$ \textsf{dom(}$I_0$\textsf{)}
~\\~\\
(1) $I_0$ \textsf{with}$\#m$\textsf{(}$I\ \_$\textsf{val);} $\in$ \textsf{refine(}$I_0,\overline{meth}$\textsf{)} \textcolor{red}{(part I definition of newChilds) fields with underscore, field and isField}
    \begin{itemize}
    \item \textsf{isWith(mbody(with}$\#m,I_0$\textsf{)}$,I_0$\textsf{)}
    \item \textsf{with}$\#m$ $\notin$ \textsf{dom(}$\overline{meth}$\textsf{)}
    \end{itemize}
(2) $I_0\ \_m$\textsf{(}$I\ \_$\textsf{val);} $\in$ \textsf{refine(}$I_0,\overline{meth}$\textsf{)} \textcolor{red}{(part II definition of newChilds)}
    \begin{itemize}
    \item \textsf{isSetter(mbody(}$\_m,I_0$\textsf{)}$,I_0$\textsf{)}
    \item $\_m$ $\notin$ \textsf{dom(}$\overline{meth}$\textsf{)}
    \end{itemize}
(3) \textsf{valid(}$I_0$\textsf{)} if $\forall m\ \in\ $\textsf{dom(}$I_0$\textsf{)}, let $meth$ = \textsf{mbody(}$m,I_0$\textsf{)}, one of the following cases is satisfied:  \textcolor{red}{(definition of valid)}
    \begin{itemize}
    \item \textsf{isField(}$meth$\textsf{)}, where \textsf{isField(}$I\ m$\textsf{();)} = not \textsf{special(}$m$\textsf{)}
    \item \textsf{isWith(}$meth,I_0$\textsf{)}, where \textsf{isWith(}$I'$ \textsf{with}$\#m$\textsf{(}$I\ x$\textsf{);}$,I_0$\textsf{)} = $I_0 :< I'$, \textsf{mbody(}$m,I_0$\textsf{)} = $I\ m$\textsf{();} and not \textsf{special(}$m$\textsf{)}
    \item \textsf{isSetter(}$meth,I_0$\textsf{)}, where \textsf{isSetter(}$I'$ $\_m$\textsf{(}$I\ x$\textsf{);}$,I_0$\textsf{)} = $I_0 :< I'$, \textsf{mbody(}$m,I_0$\textsf{)} = $I\ m$\textsf{();} and not \textsf{special(}$m$\textsf{)}
    \end{itemize}
(4) \textsf{ofMethod(}$I_0$\textsf{)} = \textsf{static }$I_0$\textsf{ of(}$I_1\ \_m_1,\cdots,I_n\ \_m_n$\textsf{) \{} \textcolor{red}{(definition of ofMethod)}
    \\ \textsf{return new }$I_0$\textsf{() \{}
    \\ $I_1\ m_1$ = $\_m_1$\textsf{;}$\cdots I_n\ m_n$ = $\_m_n$\textsf{;}
    \\ $I_1\ m_1$\textsf{() \{return }$m_1$\textsf{;\}}$\cdots I_n\ m_n$\textsf{() \{return }$m_n$\textsf{;\}}
    \\ \textsf{withMethod(}$I_1,m_1,I_0,\overline{e}_1$\textsf{)}$\cdots$\textsf{withMethod(}$I_n,m_n,I_0,\overline{e}_n$\textsf{)}
    \\ \textsf{setterMethod(}$I_1,m_1,I_0$\textsf{)}$\cdots$\textsf{setterMethod(}$I_n,m_n,I_0$\textsf{)}
    \\ \textsf{\};\}}
    \begin{itemize}
    \item $I_1\ m_1$\textsf{();}$\cdots I_n\ m_n$\textsf{();} = \textsf{fields(}$I_0$\textsf{)}
    \item $\overline{e}_i$ = $m_1,\cdots,m_{i-1},\_$\textsf{val}$,m_{i+1},\cdots,m_n$
    \end{itemize}
(5) $meth$ $\in$ \textsf{fields(}$I_0$\textsf{)} \textcolor{red}{(definition of fields)}
    \begin{itemize}
    \item \textsf{isField(}$meth$\textsf{)}
    \item $meth$ = \textsf{mbody(}$m^{meth},I_0$\textsf{)}
    \end{itemize}
(6) $I_0$ \textsf{with}$\#m$\textsf{(}$I\ \_$\textsf{val) \{return }$I_0$\textsf{.of(}$\overline{e}$\textsf{);\}} = \textsf{withMethod(}$I,m,I_0,\overline{e}$\textsf{)} \textcolor{red}{(definition of withMethod)}
    \begin{itemize}
    \item \textsf{mbody(with}$\#m,I_0$\textsf{)} has the form \textsf{mh;}
    \end{itemize}
(7) $I_0$ $\_m$\textsf{(}$I\ \_$\textsf{val) \{}$m$ = $\_$\textsf{val;return this;\}} = \textsf{setterMethod(}$I,m,I_0$\textsf{)} \textcolor{red}{(definition of setterMethod)}
    \begin{itemize}
    \item \textsf{mbody(}$\_m,I_0$\textsf{)} has the form \textsf{mh;}
    \end{itemize}
\caption{Translation of \lstinline{@Obj} and \lstinline{@ObjOf}.}
\label{fig:trans2}
\end{figure*}

\section{Implementation}\label{sec:implementation}

\section{Case Study}\label{sec:casestudy}

\subsection{A Prettier Printer}

This case study refactors the Haskell code from \cite{Wadler98aprettier}
which uses deep embeddings to implement a functional pretty printer library with high efficiency. In the original code, two
data structures are defined for documents:
\begin{lstlisting}[keywords={data,Int,String}]
data DOC = NIL
            | DOC :<> DOC
            | NEST Int DOC
            | TEXT String
            | LINE
            | DOC :<> DOC
data Doc = Nil
            | String `Text` Doc
            | Int `Line` Doc
\end{lstlisting}
Here they are encoded with shallow DSLs, and packaged into
two base families with \textsf{@Family} annotation applied. These two families \textsf{Family\_Doc} and
\textsf{Family\_Document} correspond to data types \textsf{Doc} and \textsf{DOC} respectively.

\begin{lstlisting}
@Family interface Family_Doc {
	interface Doc {}
	interface Nil extends Doc {}
	interface Text extends Doc {
		String _s(); Doc _d();
	}
	interface Line extends Doc { int _i(); Doc _d(); }
}

@Family interface Family_Document {
	interface Document {}
	interface DNil extends Document {}
	interface DConcat extends Document {
		Document _d1(); Document _d2();
	}
	interface DNest extends Document {
		int _i(); Document _d();
	}
	interface DText extends Document { String _s(); }
	interface DLine extends Document {}
	interface DUnion extends Document {
		Document _d1(); Document _d2();
	}
}
\end{lstlisting}

With the help of family polymorphism and \textsf{@Family}, we can easily add new operations and integrate them
in child families. Below is an example of encoding two original functions \textsf{layout} and \textsf{fits} in
the shallow embeddings. Note that with the annotation processing, verbose code for building inheritance relations is
automatically generated.
\begin{lstlisting}
@Family interface Family_Doc_LayoutFits extends Family_Doc {
	interface Doc {
		String layout(); boolean fits(int w);
	}
	interface Nil {
		default String layout() { return ""; }
		default boolean fits(int w) {
			return w >= 0;
		}
	}
	...
\end{lstlisting}
Furthermore, the function \textsf{be} needs refactoring before it is encoded in shallow embeddings.
The original \textsf{be} is defined as follows:
\begin{lstlisting}
be w k [] = Nil
be w k ((i,NIL):z) = be w k z
be w k ((i,x :<> y):z) = be w k ((i,x):(i,y):z)
be w k ((i,NEST j x):z) = be w k ((i+j,x):z)
be w k ((i,TEXT s):z) = s `Text` be w (k+length s) z
be w k ((i,LINE):z) = i `Line` be w i z
be w k ((i,x :<|> y):z) = better w k (be w k ((i,x):z)) (be w k ((i,y):z))
\end{lstlisting}
We refactor it by introducing a helper function \textsf{beaux}:
\begin{lstlisting}
beaux w k i NIL z = be w k z
beaux w k i (x :<> y) z = be w k ((i,x):(i,y):z)
beaux w k i (NEST j x) z = be w k ((i+j,x):z)
beaux w k i (TEXT s) z = s `Text` be w (k+length s) z
beaux w k i LINE z = i `Line` be w i z
beaux w k i (x :<|> y) z = better w k (be w k ((i,x):z)) (be w k ((i,y):z))

be w k [] = Nil
be w k ((i,x):z) = beaux w k i x z
\end{lstlisting}
In this case, \textsf{beaux} can be defined as an operation method \lstinline{beaux(int, int, int, List<Pair<int, Document>>)} inside the type \textsf{Document}, and
its pattern matching corresponds to the default implementations in different member types (like \textsf{DNil}, \textsf{DConcat}, etc). On the other hand, \textsf{be} is implemented as a static method
\lstinline{be(int, int, List<Pair<int, Document>>)} outside the member types.

Finally we finished the refactoring of the pretty printer library using a set of extensible families with shallow DSLs and operations, and our
\textsf{@Family} annotation. However there is a special operation called \textsf{group}, which works like a transformation:
\begin{lstlisting}
@Family interface Family_Document_Flatten extends Family_Document {
	interface Document {
		Document flatten();
		default Document group() {
			return DUnion.of(this.flatten(), this);
		}
	}
	...
}
\end{lstlisting}
The method \textsf{group} calls the constructor method from \textsf{DUnion}, whereas such a member type will be updated in the
child families, together with its constructors. To ensure type safety we are unable to reuse the code but just copy it and paste
to the child families, and hence code duplication is introduced by the use of factory methods. The matter of extensible
transformations will be further discussed in the next section.



\section{Related Work}\label{sec:relatedwork}

\paragraph{Family Polymorphism}

There has been a lot of related work on family polymorphism~\cite{ernst2001family}, including various lightweight encodings~\cite{Kamina:2007:LSC:1289971.1289996,saito2007essence,igarashi2005lightweight,Kamina:2008:LDC:1449913.1449932}. In the original
paper~\cite{ernst2001family}, nested classes are represented as attributes of an object, which involves a dependent type system. In~\cite{igarashi2005lightweight}, Saito and Igarashi
proposed a lightweight variant of family polymorphism, which uses classes rather than objects to represent families. Later work in~\cite{igarashi2005lightweight}
proposes a simple extension to Featherweight Generic Java~\cite{Igarashi:2001:FJM:503502.503505}, introducing self-type variables. There are also some existing languages,
like Scala, which supports symmetric mixin compositions and self-type annotations[?], hence can provide better support for encoding family
polymorphism. Our lightweight encoding relies entirely on the existing Java language without language extensions, and certainly some features from
the original paper are sacrificed, including the mismatching problem of recursive class definitions (also known as binary methods~\cite{bruce1995binary}). Relying
on the Java semantics, covariant return types are supported and used for our automatic type refinements. Nevertheless, in the case of binary methods,
we cannot address those with recursive member types as parameters. This problem is however handled in some other work like MyType[?], ThisType[?] and self-type
annotations in Scala as we said.

On the other hand, our approach uses nested interfaces to build the relationship among families and members, which is different from
path-dependent types in~\cite{ernst2001family}. The work in~\cite{ernst2003higher} inspires us with the concept of higher-order hierarchies, and hence our \textsf{@Family} annotation
provides support for nested families. Furthermore, our lightweight encoding of family polymorphism is polished with the help of annotation processing
to generate constructor methods automatically.

%[1] family polymorphism
%[2] Lightweight scalable components
%[3] The essence of lightweight family polymorphism
%[4] Lightweight family polymorphism
%[5] Lightweight dependent classes
%[6] Featherweight Java
%[7] On binary method
%[8] Higher-order hierarchies

%\paragraph{ThisType}

\paragraph{Multiple Inheritance}

Multiple inheritance is an expressive and useful feature in programming languages, yet difficult to model in relation to the famous diamond problem[?].
Many models have been developed to integrate multiple inheritance in some languages, including C++ virtual inheritance[?], mixins[?], traits[?] and
hybrid models such as CZ[?]. In our approach, introducing new variants or operations requires subtyping relations among member types to be built, and
hence we rely on the Java multiple interface inheritance. A difference from the other models is that our \textsf{@Family} annotation can check method
typing and detect conflicts during the annotation processing, and on the other hand, subtyping relations are automatically generated for the users, which
makes client code more concise. Furthermore, we have mentioned that interfaces are updated with auto-generated constructor methods, whereas object initialization is a big problem in some other models.


\section{Conclusion}\label{sec:conclusion}

% Discussion and Limitations

%Haoyuan writes this one.

%- Transformations and Binary methods (but argue that
%these types of operations are not used in shallow DSLs
%any way!).

%- Java syntax not great for DSLs.
%operator overloding; infix operators (but this is a Java limitation;
%not a limitation of our approach).

%- Limitations of the tool: separate compiltions and ...
%generics not fully implemented.

%- How to apply our approach to other OO language
%  - C\# has annotations, but can you do AST rewriting?
%  - Scala has macros;
%  - Other languages?



\paragraph{Transformations} Our approach tends to give a taste that shallow embeddings are convenient to
use in object-oriented languages, and extensible with family polymorphism. Nevertheless, shallow embeddings are also under
criticism in terms of its not supporting ASTs explicitly, leading to the fact that some operations like transformations
cannot be easily used with shallow embeddings. In Section ?(2.3) we give an example of desugaring in Java using shallow
embeddings, however when we try to apply family polymorphism for inheritance and extensibility, code duplication can be introduced.
Suppose we firstly integrate the four members \textsf{Circuit}, \textsf{Fan}, \textsf{Beside} and \textsf{Identity} in a base family,
and then define a new family that extends the base one, then inside the new member \textsf{Identity}, how can we implement the transformation \textsf{desugar()}? Certainly we cannot use \textsf{super} to call the old implementation, otherwise we are invoking old constructors of \textsf{Beside},
leading to type errors afterwards. For sure we can just copy and paste the code for the old implementation, but too much
code duplication would depress the users. Another solution is to use F-bounded polymorphism (\textcolor{red}{Haoyuan: refs. Need example?
We do have an example in the Pretty Printer case study}),
in that case code duplication is avoided, but accordingly, it arouses explosively heavy uses of types and parameterisations.

\paragraph{Binary methods} Binary methods are indeed another sort of operations that prevent users from using shallow embeddings in Java
comfortably. \textcolor{red}{Haoyuan: refs?} Java supports covariant return types, hence we can refine the types of field methods and
simple interpretations, but with binary methods like \textsf{equals}, method subtyping in inheritance is a big problem since the argument type
is changed. In Scala, however, we can deal with binary methods with the help of ? \textcolor{red}{Haoyuan: Could you help Weixin?}.

\paragraph{Shallow or deep embeddings?} We know the fact that our approach integrates shallow embeddings and simple family polymorphism, and that in some object-oriented languages like Java,
some trouble is introduced when dealing with extensible transformations or binary methods. But on the other hand, these types of operations are not used in
shallow DSLs anyway; with a single interpretation we use shallow embeddings, but since shallow embeddings do not represent abstract syntax trees,
it is really hard to realize transformations, and similarly, with binary methods, users tend to use deep embeddings instead to build data hierarchies.

\paragraph{Limitations} Our tool has its certain limitations at a few aspects. As Lombok only provides experimental support
for separate compilation, our \textsf{@Family} does not support separate compilation at this stage, so all the related
interfaces should be included in a single Java file. On generics, our implementation of \textsf{@Family} provides support
for generic interfaces and methods without bounds, on the other hand, generic method typing is not explicitly captured
by the annotation but delegated to the Java compiler. \textcolor{red}{Haoyuan: We still need to talk about Lombok in detail?
Another limitation is that only Eclipse handler is supported.}

Regarding syntax, it would be nice to have operator overloading and infix operators in Java, so that the code could be
written in a more concise and elegant way. Nevertheless, this is limited by the programming language, and we expect our approach to be oriented to
other OO languages. \textcolor{red}{Haoyuan: How to apply our approach to other OO language? C\# has annotations, but can you do AST rewriting? Scala has macros;  Other languages?}



\begin{comment}
\appendix
\section{Appendix Title}

This is the text of the appendix, if you need one.

\acks

Acknowledgments, if needed.
\end{comment}

% The 'abbrvnat' bibliography style is recommended.

% The bibliography should be embedded for final submission.

%\begin{thebibliography}{}
%\softraggedright

%\bibitem[Smith et~al.(2009)Smith, Jones]{smith02}
%P. Q. Smith, and X. Y. Jones. ...reference text...

%\end{thebibliography}

\newpage
\clearpage
\bibliographystyle{abbrvnat}
\bibliography{paper}

\end{document}

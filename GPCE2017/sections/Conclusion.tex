\section{Conclusion}\label{sec:conclusion}

% Discussion and Limitations

%Haoyuan writes this one.

%- Transformations and Binary methods (but argue that
%these types of operations are not used in shallow DSLs
%any way!).

%- Java syntax not great for DSLs.
%operator overloding; infix operators (but this is a Java limitation;
%not a limitation of our approach).

%- Limitations of the tool: separate compiltions and ...
%generics not fully implemented.

%- How to apply our approach to other OO language
%  - C\# has annotations, but can you do AST rewriting?
%  - Scala has macros;
%  - Other languages?



\paragraph{Transformations} Our approach tends to give a taste that shallow embeddings are convenient to
use in object-oriented languages, and extensible with family polymorphism. Nevertheless, shallow embeddings are also under
criticism in terms of its not supporting ASTs explicitly, leading to the fact that some operations like transformations
cannot be easily used with shallow embeddings. In Section ?(2.3) we give an example of desugaring in Java using shallow
embeddings, however when we try to apply family polymorphism for inheritance and extensibility, code duplication can be introduced.
Suppose we firstly integrate the four members \textsf{Circuit}, \textsf{Fan}, \textsf{Beside} and \textsf{Identity} in a base family,
and then define a new family that extends the base one, then inside the new member \textsf{Identity}, how can we implement the transformation \textsf{desugar()}? Certainly we cannot use \textsf{super} to call the old implementation, otherwise we are invoking old constructors of \textsf{Beside},
leading to type errors afterwards. For sure we can just copy and paste the code for the old implementation, but too much
code duplication would depress the users. Another solution is to use F-bounded polymorphism (\textcolor{red}{Haoyuan: refs. Need example?
We do have an example in the Pretty Printer case study}),
in that case code duplication is avoided, but accordingly, it arouses explosively heavy uses of types and parameterisations.

\paragraph{Binary methods} Binary methods are indeed another sort of operations that prevent users from using shallow embeddings in Java
comfortably. \textcolor{red}{Haoyuan: refs?} Java supports covariant return types, hence we can refine the types of field methods and
simple interpretations, but with binary methods like \textsf{equals}, method subtyping in inheritance is a big problem since the argument type
is changed. In Scala, however, we can deal with binary methods with the help of ? \textcolor{red}{Haoyuan: Could you help Weixin?}.

\paragraph{Shallow or deep embeddings?} We know the fact that our approach integrates shallow embeddings and simple family polymorphism, and that in some object-oriented languages like Java,
some trouble is introduced when dealing with extensible transformations or binary methods. But on the other hand, these types of operations are not used in
shallow DSLs anyway; with a single interpretation we use shallow embeddings, but since shallow embeddings do not represent abstract syntax trees,
it is really hard to realize transformations, and similarly, with binary methods, users tend to use deep embeddings instead to build data hierarchies.

\paragraph{Limitations} Our tool has its certain limitations at a few aspects. As Lombok only provides experimental support
for separate compilation, our \textsf{@Family} does not support separate compilation at this stage, so all the related
interfaces should be included in a single Java file. On generics, our implementation of \textsf{@Family} provides support
for generic interfaces and methods without bounds, on the other hand, generic method typing is not explicitly captured
by the annotation but delegated to the Java compiler. \textcolor{red}{Haoyuan: We still need to talk about Lombok in detail?
Another limitation is that only Eclipse handler is supported.}

Regarding syntax, it would be nice to have operator overloading and infix operators in Java, so that the code could be
written in a more concise and elegant way. Nevertheless, this is limited by the programming language, and we expect our approach to be oriented to
other OO languages. \textcolor{red}{Haoyuan: How to apply our approach to other OO language? C\# has annotations, but can you do AST rewriting? Scala has macros;  Other languages?}


\section{Overview}\label{sec:overview}

In this section, we illustrate how to model simple family polymorphism in Java with interfaces,
and how \name avoids boilerplate code for more usability.

\subsection{Modelling Families with Nested Interfaces}

The original family polymorphism paper [] and higher-order hierarchies paper [] motivates the need for class families. Software programming
often requires ``the variability at a more global scale than the individual class''. For example, a data
structure is usually represented by an abstract syntax tree, and such an AST may contain several classes.
It is necessary to maintain the consistency of those classes upon updates or extensions. \red{TODO: Move this paragraph
to Introduction.}

An approach to family polymorphism should ensure both type safety and code reuse. It is actually quite tricky to model
family polymorphism in existing languages like Java, without additional syntax and semantics support. Yet a sub-problem of
family polymorphism is encapsulation, namely how to group family members into modules. This motivates us to use nested components.
By Java's semantics, inner classes are dependent from their enclosing type. In contrast, nested interfaces are only static, that is
to say, they are only grouped by a module, hence it is no different to put them outside the containing interface. Obviously
the advantage is that we can keep the names of family members when they are extended. 

Now one may observe that the use of interfaces has a trade-off: multiple interface inheritance supported by Java makes it convenient for extensibility, nevertheless,
interfaces are weaker than classes on state and instantiation. In Java, interfaces do not own fields or constructors.
Declaring one class for each interface is cumbersome and introduces boilerplate.
Fortunately, we can adopt object interfaces from the Classless Java model []. In that case, fields are modelled by abstract methods,
and static methods can be defined with anonymous classes that deals with object creation.

Here we illustrate with the example from [], about parallel prefix circuits. Below we present a simplified version
of the AST, by defining a base family with only four members:

\begin{lstlisting}
interface Base {
  interface Circuit {
    int width();
  }
  interface Identity extends Circuit {
    int _size();
    default int width() { return _size(); }
    static Identity of(int _size) {
      return new Identity() {
        int size = _size;
        public int _size() { return size; }
      };
    }
  }
  interface Above extends Circuit {
    Circuit _c1(); Circuit _c2();
    default int width() { return _c1().width(); }
    static Above of(Circuit _c1, Circuit _c2) {
      return new Above() {
        Circuit c1 = _c1, c2 = _c2;
        public Circuit _c1() { return c1; }
        public Circuit _c2() { return c2; }
      };
    }
  }
  interface Beside extends Circuit {
    Circuit _c1(); Circuit _c2();
    default int width() {
      return _c1().width() + _c2().width();
    }
    static Beside of(Circuit _c1, Circuit _c2) {
      return new Beside() {
        Circuit c1 = _c1, c2 = _c2;
        public Circuit _c1() { return c1; }
        public Circuit _c2() { return c2; }
      };
    }
  }
}
\end{lstlisting}\red{Need a figure for circuits.}
Here \lstinline|Identity|, \lstinline|Above| and \lstinline|Beside| are three constructs of \lstinline|Circuit|.
\lstinline|Identity| holds an integer for the size, while \lstinline|Above| and \lstinline|Beside| each takes
two sub-circuits and combine them. Those fields are defined as ``field methods'', whose names
start with underscores by convention. Java 8 allows static methods defined in interfaces with default implementations.
The \lstinline|of()| methods behave like constructors, they are boilerplate code that can be generated. We inherit from
Classless Java to support a series of state operations. They are formally introduced in Section ?.

Besides, an operation to calculate the width of a circuit is declared in \lstinline|Circuit|. In the other members such
a method is given a default implementation, respectively.

Note that here \lstinline|Base| is the family name, but since the nested interfaces are only static, one
can easily access them globally by \lstinline|Base.Circuit|, and so on. For example, one can create a circuit instance and
quickly get its width as follows:
\begin{lstlisting}
Base.Circuit c = Base.Beside.of(Base.Identity.of(2), Base.Identity.of(3));
println(c.width()); // "5"
\end{lstlisting}

\subsection{Solving the Expression Problem}

Interestingly but also importantly, it introduces a new solution to the Expression Problem. Adding new variants and new operations
becomes easy, and inheritance contributes a lot to code reuse. Following the above example, we try to add a new operation to check
if a circuit is well-sized. Such an operations is dependent on the existing \lstinline|width()| operation.
A new class family is defined as follows:

\begin{lstlisting}
interface WellSized extends Base {
  interface Circuit extends Base.Circuit {
    boolean wellSized();
  }
  interface Identity extends Circuit with Base.Identity {
    // int _size(); not needed
    default boolean wellSized() { return true; }
    static Identity of(...) {...}
  }
  interface Above extends Circuit with Base.Above {
    Circuit _c1(); Circuit _c2();
      // Type-refinement from Base.Circuit to BaseWithWellSized.Circuit
    default boolean wellSized() {
      return _c1().width() == _c2().width();
    }
    static Above of(...) {...}
  }
  interface Beside extends Circuit with Base.Beside {
    Circuit _c1(); Circuit _c2();
      // Type-refinement
    default boolean wellSized() {
      return _c1().wellSized() && _c2().wellSized();
    }
    static Beside of(...) {...}
  }
}
\end{lstlisting}
Here each family member also inherits its corresponding member in \lstinline|Base|, so that \lstinline|WellSized.Circuit|
supports both \lstinline|width()| and \lstinline|wellSized()|. The type-refinement of fields in \lstinline|Above| and \lstinline|Beside|
ensures that the sub-circuits support both operations, and this is done because Java allows covariant return types.
At this moment it is not necessary to make \lstinline|WellSized| as a subtype of
\lstinline|Base|, we just write this to indicate the relation between class families. The static constructors are again boilerplate
that can be generated, so we omit the details.

Unlike traditional family polymorphism, when we use inheritance to achieve code reuse, it also builds subtyping relations
among members from different families, because subclassing accompanies subtyping in Java. Separating them (if we want) requires
additional semantics, it is not cheap. Also in this case, we can no longer avoid mixing the members from different families.
In fact, isolating families or avoiding mixing the members is sometimes too conservative; it has been shown that mixing them can
be both type-safe and practical []. On the other hand, one can also check the dynamic types of objects to ensure they belong to
the same family, and this can simply be done by some Java reflection code.

By the Expression Problem, the other dimension of extensibility is adding new data constructs. Suppose we want another primitive
construct \lstinline|Fan| to be added
to the \lstinline|Base| family. The following code realizes such an extension:

\begin{lstlisting}
interface BaseFan extends Base {
  interface Circuit extends Base.Circuit {}
  interface Identity extends Circuit with Base.Identity {
    static Identity of(...) {...}
  }
  interface Above extends Circuit with Base.Above {
    Circuit _c1(); Circuit _c2();
    static Above of(...) {...}
  }
  interface Beside extends Circuit with Base.Beside {
    Circuit _c1(); Circuit _c2();
    static Beside of(...) {...}
  }
  interface Fan extends Circuit {
    int _size();
    default int width() { return _size(); }
    static Identity of(...) {...}
  }
}
\end{lstlisting}
Here \lstinline|Circuit|, \lstinline|Identity|, \lstinline|Above| and \lstinline|Beside| only deal with type-refinements and inheritance. A
new nested interface \lstinline|Fan| includes one field method for its size, together with
the implementation for the inherited \lstinline|width()|.

We have demonstrated that such a pattern provides a practical solution to the Expression Problem. The member
names need not to be changed, and multiple inheritance offers great code reuse. However, one issue occurring
in most object-oriented languages is that when recursive types appear in the parameters or return types,
such methods violate extensibility. They are the well-known \textit{binary methods} []. Some solutions to binary methods
like \textit{ThisType} or \textit{MyType} [] again require language support. A workaround to this in Java
is the \textit{F-bounded polymorphism} [], which we will discuss later in Section ?.

\subsection{Reducing Boilerplate with \name}
Although the class families support two dimensions of extensibility, it is unsatisfactory to observe that there is a lot
of boilerplate code caused by extensions, and it can be classified into three kinds: the constructor methods, the glue code for subtyping
and field type refinements. The first one has been addressed by annotation processing in Classless Java. Besides,
the glue code builds subtyping relations among families and their members, so that fields and operations are inherited.
And the type-refinement code updates the field types to those in the same family, and this is done by just redeclaring field
methods (but member types are already new).

We have developed the Java annotation \lstinline|@Family| to reduce all the three sorts of boilerplate. The annotation processing can automatically
generate the required syntax in bytecode during compilation, and does not affect source code. With our framework, \lstinline|WellSized|
and \lstinline|BaseFan| can be defined by users as follows:
\begin{lstlisting}
@Family interface WellSized extends Base {
  interface Circuit {
    boolean wellSized();
  }
  interface Identity {
    default boolean wellSized() { return true; }
  }
  interface Above {
    default boolean wellSized() {
      return _c1().width() == _c2().width();
    }
  }
  interface Beside {
    default boolean wellSized() {
      return _c1().wellSized() && _c2().wellSized();
    }
  }
}

@Family interface BaseFan extends Base {
  interface Fan extends Circuit {
    int _size();
    default int width() { return _size(); }
  }
}
\end{lstlisting}
Here \lstinline|@Family| automatically fills in the super types of family members, and meanwhile
refines field types. Users only need to write the interesting code of extensions. The above code
also explains why we need subtyping among families, because it indicates what the annotation should do.

The family subtyping also achieves independent extensibility []. Suppose we have a list of families,
each introducing several new operations on expressions. By multiple inheritance, we can easily compose
those features without boilerplate, like the code below:
\begin{lstlisting}
interface Width {...} // the family with width operation
interface Depth {...} // the family with depth operation
...

@Family interface ComposedFamily extends Width, Depth, ... {} // combining features
\end{lstlisting}

In the above example, combining features does not need any extra code in the body. Similarly we can also
combine \lstinline|WellSized| and \lstinline|BaseFan|, nonetheless, we need to implement the
pretty-printing on subtractions, so that the object interfaces are able to be instantiated:
\begin{lstlisting}
@Family interface WellSizedFan extends WellSized, BaseFan {
  interface Fan {
    default boolean wellSized { return true; }
  }
}
\end{lstlisting}
Now we can test the new family with some code below: (the \lstinline|of()| methods are already generated by \lstinline|@Family|)
\begin{lstlisting}
WellSizedFan.Circuit c2 = WellSizedFan.Above.of(WellSizedFan.Identity.of(2), WellSizedFan.Fan.of(3));
println(c2.wellSized()); // "false"
\end{lstlisting}

In our prototype implementation, \lstinline|@Family| also supports generics and higher-order hierarchies [].
Simple generics support makes it practical to use, and higher-order hierarchies allow us to define nested families,
where annotation processing can deeply work on all the inside interfaces. Since families and members are both defined
as interfaces, and they rely on subtyping, it is just not necessary to distinguish them.





